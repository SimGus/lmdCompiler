\documentclass{article}

\usepackage[francais]{babel}
\usepackage[utf8]{inputenc}
\usepackage[T1]{fontenc}

\usepackage[normalem]{ulem}
\usepackage{enumitem}
\usepackage{graphicx}
\usepackage{hyperref}
\usepackage{textcomp}
\usepackage[hscale=0.73,vscale=0.82]{geometry}
\renewcommand*{\familydefault}{\sfdefault}

\title{Syntaxe d'un langage de \textit{markdown} à traduire en LaTeX\\\underline{Définition}\\2-6-2016\\Simon Gustin}
\author{}
\date{}
%Première ligne du titre

\begin{document}
\maketitle

	

	\part{\underline{Caractères}}

		\section{\textbf{Typographie}}

			Texte normal

			\textit{Texte italique}

			\textbf{Texte gras}

			\underline{Texte souligné}

			\sout{Texte barré}

			\emph{Texte important}

			\textrm{ Citation (})

			

		\section{\textit{Caractères spéciaux}}%comment

			À n'utiliser que si la compilation les interprète comme du markdown

			\textbackslash  backslash

			\_ underscore

			\# croisillon

			$\sim$ tilde

			etc.

			

			... pour ne pas remplacer \dots  par …

			

			\$ @ € ° | \& £ µ [ ] \{\} \%%

			

	\part{\underline{Mise en page}}

		Paragraphe

		Passage à la ligne

		

		\begin{figure}[h!]
			\centering
			\includegraphics{image.png}
			\caption{label}
		\end{figure}

		\url{http://link.com}

		\href{http://link.com/}{label}

		\verb?Plain text % $ #include<stdlib.h> int main(){return EXIT_SUCCESS;}?

		

\begin{verbatim}
Plain text $^? test\ @&%
\end{verbatim}
		

		\textendash \textendash 

		|Tableau|cellule 2|

		\textendash \textendash (\textendash \textendash \textendash \textendash \textendash \textendash \textendash \textendash \textendash \textendash \textendash \textendash \textendash \textendash \textendash )

		

		%commentaire

		

		+\textendash \textendash \textendash \textendash \textendash \textendash \textendash \textendash \textendash +\textendash \textendash \textendash \textendash \textendash \textendash \textendash \textendash \textendash \textendash \textendash \textendash \textendash \textendash +

		|Titre tableau spécifique|

		+\textendash \textendash \textendash \textendash \textendash \textendash \textendash \textendash \textendash +\textendash \textendash \textendash \textendash \textendash \textendash \textendash \textendash \textendash \textendash \textendash \textendash \textendash \textendash +

		|cellule  |cellule       |

		+\textendash \textendash \textendash \textendash \textendash \textendash \textendash \textendash \textendash +\textendash \textendash \textendash \textendash \textendash \textendash \textendash \textendash \textendash \textendash \textendash \textendash \textendash \textendash +

		

		\section{Énumérations}

			\begin{itemize}[label=$\bullet$]
				\item item 1
				\item item 2
					\begin{itemize}[label=$\bullet$]
						\item item 2.a
						\item item 2.b
					\end{itemize}
			\end{itemize}

			

enum : 1
			2. liste 2

			\begin{itemize}[label=$\bullet$]
				\item item 2.a
				\item item 2.b
			\end{itemize}

enum : a
			b. liste 2.b.2

			

	\part{Partie}

		\section{Section}

			\subsection{Subsection}

				\subsubsection{Subsubsection}

					\paragraph{Paragraphe}%avec commentaire
						Avec quelque chose dedans

						\subparagraph{Sous-paragraphe}%com
							Test

\end{document}

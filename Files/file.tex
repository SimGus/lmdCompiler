\documentclass{article}

\usepackage[francais]{babel}
\usepackage[utf8]{inputenc}
\usepackage[T1]{fontenc}

\usepackage[normalem]{ulem}
\usepackage{enumitem}
\usepackage{graphicx}
\usepackage{hyperref}
\usepackage{spverbatim}
\usepackage{textcomp}
\usepackage[hscale=0.73,vscale=0.82]{geometry}
\renewcommand*{\familydefault}{\sfdefault}

\title{Syntaxe d'un langage de \textit{markdown} à traduire en \textbf{LaTeX}\\\underline{Définition}\\2-6-2016\\Simon Gustin}
\author{}
\date{}
%Première ligne du titre

\begin{document}
\maketitle

	

	\part{\underline{Caractères}}

		\section{\textbf{Typographie}}

			Texte normal

			\textit{Texte italique}

			\textbf{Texte gras}

			\underline{Texte souligné}

			\sout{Texte barré}

			\emph{Texte important}

			\textrm{Citation }

			Not interpreted text : \LaTeX{}

			

		\section{\textit{Caractères particuliers}}%comment

			À n'utiliser que si la compilation les interprète comme du markdown

			\textbackslash  backslash

			\_ underscore

			\# croisillon

			$\sim$ tilde

			etc.

			

			... pour ne pas remplacer \dots  par …

			

			\$ @ € ° | \& £ µ [ ] \{\} \%%

			

	\part{\underline{Mise en page}}

		Paragraphe

		Passage à la ligne

		

		\begin{figure}[h!]
			\centering
			\includegraphics[scale=10.000000]{image.png}
			\caption{label | de l'\textbf{image} }
		\end{figure}

		\url{http://link.com}

		\href{http://link.com/}{label à plusieurs \textit{mots}}

		\verb?Plain text % $ #include<stdlib.h> int main(){return EXIT_SUCCESS;}?

		

\begin{spverbatim}
Plain text $^? test\ @&%
\end{spverbatim}
		

		\begin{tabular}{|l|l|}
			\hline
			\textbf{Tableau} & cellule 2\\
			\hline
			Ligne 2 & Pas de ligne horizontale\\
			Ligne 3 & \textit{vide}\\
			\hline
		\end{tabular}

		

		%commentaire

		

		+=========+==============+

		|Titre tableau spécifique|

		+\textendash \textendash \textendash \textendash \textendash \textendash \textendash \textendash \textendash +\textendash \textendash \textendash \textendash \textendash \textendash \textendash \textendash \textendash \textendash \textendash \textendash \textendash \textendash +

		|cellule  |cellule       |

		+=========+==============+

		

		\section{Énumérations with \texttt{Teletype}}

			\begin{itemize}[label=$\bullet$]
				\item item 1
				\item item 2
					\begin{itemize}[label=$\bullet$]
						\item item 2.a
						\item item 2.b
					\end{itemize}
			\end{itemize}

			

			\begin{enumerate}
				\item liste 1
				\item liste 2
					\begin{itemize}[label=$\bullet$]
						\item item 2.a with \textit{italic}
						\item item 2.b
							\begin{enumerate}
								\item liste 2.b.1 with \verb?plain text?
								\item liste 2.b.2
							\end{enumerate}
						\item item 2.c
					\end{itemize}
				\item liste 3
			\end{enumerate}

			

	\part{Partie}

		\section{Section}

			\subsection{Subsection}

				\subsubsection{Subsubsection}

					\paragraph{Paragraphe}%avec commentaire
						Avec quelque chose dedans

						\subparagraph{Sous-paragraphe}%com
							Test

\end{document}
